\documentclass{article}
\usepackage{polski}
\usepackage[utf8]{inputenc}
\usepackage{csvsimple}
\usepackage{hyperref}
\usepackage{enumitem}
\hypersetup{
    colorlinks,
    linkcolor=[rgb]{.0 .152 .645}
}
\usepackage{enumitem}
\usepackage{listings}
\usepackage[edges]{forest}
\definecolor{folderbg}{RGB}{124,166,198}
\definecolor{folderborder}{RGB}{110,144,169}
\newlength\Size
\setlength\Size{4pt}
\tikzset{%
  folder/.pic={%
    \filldraw [draw=folderborder, top color=folderbg!50, bottom color=folderbg] (-1.05*\Size,0.2\Size+5pt) rectangle ++(.75*\Size,-0.2\Size-5pt);
    \filldraw [draw=folderborder, top color=folderbg!50, bottom color=folderbg] (-1.15*\Size,-\Size) rectangle (1.15*\Size,\Size);
  },
  file/.pic={%
    \filldraw [draw=folderborder, top color=folderbg!5, bottom color=folderbg!10] (-\Size,.4*\Size+5pt) coordinate (a) |- (\Size,-1.2*\Size) coordinate (b) -- ++(0,1.6*\Size) coordinate (c) -- ++(-5pt,5pt) coordinate (d) -- cycle (d) |- (c) ;
  },
}
\forestset{%
  declare autowrapped toks={pic me}{},
  pic dir tree/.style={%
    for tree={%
      folder,
      font=\ttfamily,
      grow'=0,
    },
    before typesetting nodes={%
      for tree={%
        edge label+/.option={pic me},
      },
    },
  },
  pic me set/.code n args=2{%
    \forestset{%
      #1/.style={%
        inner xsep=2\Size,
        pic me={pic {#2}},
      }
    }
  },
  pic me set={directory}{folder},
  pic me set={file}{file},
}

\title{\textbf{Współbieżny algorytm Brandesa}}
\author{Mikołaj Walczak}
\date{\vspace{-5ex}}

\begin{document}
  \newcommand{\HRule}{\rule{\linewidth}{0.5mm}} % Defines a new command for the horizontal lines, change thickness here
  \pagenumbering{gobble}
  \begin{titlepage}
    \center
    \LARGE Uniwersytet Warszawski\\[0.1cm]
    \Large Wydział Matematyki Informatyki i Mechaniki\\[0.5cm] % Major heading such as course name
    \large Programowanie Współbieżne\\[6.1cm] % Minor heading such as course title
    { \huge \bfseries Algorytm Brandesa}\\[0.5cm] % Title of your document
    \large \emph{Autor:}\\ Mikołaj Walczak \\
    \vspace{\fill}
    \begin{minipage}[b]{\textwidth}
        \centering
        \large Rok Akademicki 2016/2017

        \vspace{-20mm}
    \end{minipage}%
  \end{titlepage}

  \tableofcontents
  {%
  \let\oldnumberline\numberline%
  \renewcommand{\numberline}{\figurename~\oldnumberline}%
  \listoffigures%
  }
  {%
  \let\oldnumberline\numberline%
  \renewcommand{\numberline}{\tablename~\oldnumberline}%
  \listoftables%
  }
  \newpage

  \pagenumbering{arabic}
  \section{Implementacja}
    Poniższy rozdział zawiera opis implementacji. Szczegóły dotyczące klas i
    metod przez nie udostępnianych są dostępne w dokumentacji technicznej.
    \hyperref[documentation]{dokumentacji technicznej}.
    \label{implementation}
    \subsection{Podział na pliki}
      \label{files}
      Implementacja została podzielona względem odpowiedzialności za poszczególne
      części programu na pliki (\hyperref[fig:files]
      {\figurename~\ref*{fig:files}}).

      \begin{figure}[h]
        \label{fig:files}
        \begin{forest}
          pic dir tree, where level=0{}{directory,},
          [brandes
            [src
              [betweennes.h/.cpp \textrm{-- zawiera implementację algorytmu
                Brandesa}, file]
              [graph.h/.cpp \textrm{--
                klasy \hyperref[class:brandes__node]{\texttt{Node}}
                oraz \hyperref[class:brandes__graph]{\texttt{Graph}}}
              , file]
              [scheduler.h \textrm{--
                klasy \hyperref[class:synchronization__scope]{\texttt{Scope}}
                oraz \hyperref[class:synchronization__scheduler]
                {\texttt{Scheduler}}}
              , file]
              [main.cpp \textrm{-- odpowiedzialny za wykonanie programu}, file]
            ]
          ]
        \end{forest}
        \caption{Podział na pliki}
      \end{figure}

    \subsection{Podział na klasy}
      Podział na klasy przedstawiony jest na \hyperref[fig:classes]
      {Rysunku \ref*{fig:classes}}.
      \label{classes}
      \begin{figure}[h]
        \label{fig:classes}
        \begin{forest}
          for tree={
            font=\ttfamily,
            grow'=0,
            child anchor=west,
            parent anchor=south,
            anchor=west,
            calign=first,
            edge path={
              \noexpand\path [draw, \forestoption{edge}]
              (!u.south west) +(7.5pt,0) |- node[fill,inner sep=1.25pt] {} (.child anchor)\forestoption{edge label};
            },
            before typesetting nodes={
              if n=1
                {insert before={[,phantom]}}
                {}
            },
            fit=band,
            before computing xy={l=15pt},
          }
          [
            [namespace \texttt{\hyperref[namespace:brandes]{Brandes}}
              [class \texttt{\hyperref[class:brandes__node]{Node}}]
              [class \texttt{\hyperref[class:brandes__graph]{Graph}}]
              [function \texttt{\hyperref[func:brandes__calculate_weights]
                {calculate\_weights}}]
            ]
            [namespace \texttt{\hyperref[namespace:synchronization]
              {Synchronization}}
              [class \texttt{\hyperref[class:synchronization__scope]{Scope}}]
              [class \texttt{\hyperref[class:synchronization__scheduler]
                {Scheduler}}]
            ]
          ]
        \end{forest}
        \caption{Podział na klasy}
      \end{figure}

      \subsubsection{Przestrzenie nazw}
        \label{namespaces}

        \paragraph{Brandes}
        \label{namespace:brandes}
        zawiera klasy i funkcje związane bezpośrednio z implementacją algorytmu
        Brandesa.

        \paragraph{Synchronization}
        \label{namespace:synchronization}
        zawiera klasy i funkcje zapewniające współbieżność wykonywanych obliczeń.


      \subsubsection{Brandes::Node}
      \label{class:brandes__node}
      Reprezentuje pojedynczy wierzchołek w grafie skierowanym. Pozwala na
      bezpieczną wielowątkowo zmianę wagi przez operację
      \texttt{increase\_weight(weight)}.

      \subsubsection{Brandes::Graph}
      \label{class:brandes__graph}
      Reprezentuje graf skierowany. Umożliwia wczytanie sktrutury grafu ze strumienia
      \texttt{std::istream} oraz zapisanie wierzchołków wraz z wagami do strumienia
      \texttt{std::ostream}.

      \subsubsection{Brandes::calculate\_weights}
      \label{func:brandes__calculate_weights}
      Implementacja algorytmu Brandesa. Liczy wagi wierzchołków dla danego
      \hyperref[class:brandes__graph]{\texttt{grafu}} tworząc obiekt klasy
      \hyperref[class:synchronization__scheduler]{\texttt{Scheduler}} z podaną
      liczbą wątków i zleca mu współbieżne obliczenie wag osobno dla każdego
      wierzchołka. Dzięki gwarancji bezpieczeństwa metody
      \texttt{increase\_weight(weight)} klasy \hyperref[class:brandes__node]
      {\texttt{Node}} obliczenia mogą występować jednocześnie dla różnych
      wierzchołków.

      \subsubsection{Synchronization::Scope}
      \label{class:synchronization__scope}
      Klasa abstrakcyjna (właść. interface), z której musi dziedziczyć klasa
      środowiska obliczeniowego \hyperref[class:synchronization__scheduler]
      {\texttt{Scheduler}}'a. Dla danego zadania wykonuje wirtualną funkcję
      \texttt{execute(task)}.

      \subsubsection{Synchronization::Scheduler}
      \label{class:synchronization__scheduler}
      Główny mechanizm synchronizujący. Tworzy pulę wątków a w każdym z nich
      instancję podanej klasy \hyperref[class:synchronization__scope]{\texttt{Scope}}.
      Dla każdego otrzymanego zadania \texttt{task} zleca jednemu z wolnych wątków
      wykonanie funkcji \texttt{execute(task)}. Jeżeli wszystkie wątki są zajęte,
      zadania oczekują na wykonanie w kolejce do momentu zakończenia przez któryś
      z wątków obliczania poprzedniej wartości.

    \subsection{Dokumentacja techniczna}
      \label{documentation}
      Do wygenerowania dokumentacji wymagany jest pakiet \texttt{Doxygen}
      w wersji $\geq 1.8.8$. Wykonanie poniższych instrukcji w katalogu
      projektu spowoduje wygenerowanie dokumentacji w folderach jak na
      \hyperref[fig:documentation]{Rysunku \ref*{fig:documentation}}.
      \begin{verbatim}
        $ mkdir build && cd $_
        $ cmake ..
        $ make doc
      \end{verbatim}
      \vspace{-3ex}
      \begin{figure}[h]
        \label{fig:documentation}
        \begin{forest}
          pic dir tree, where level=0{}{directory,},
          [brandes
            [build
              [doc
                [html \textrm{-- dokumentacja w wersji HTML}]
                [latex \textrm{-- dokumentacja w wersji \LaTeX}]
              ]
              [Doxyfile, file]
              [Makefile, file]
            ]
            [CMakeLists.txt \textrm{-- instrukcje budowy dokumentacji}, file]
            [Doxyfile.in \textrm{-- konfiguracja dokumentacji}, file]
            [doxygen \textrm{-- pliki stylu dokumentacji HTML}]
            [src \textrm{-- kod źródłowy}]
          ]
        \end{forest}
        \caption{Generowanie dokumentacji}
      \end{figure}

    \subsection{Optymalizacje}
      \subsubsection{Współbieżność} Dzięki niezależności obliczeń wpływu pośrednictwa
      pomiędzy poszczególnymi wierzchołkami możliwe było przeniesieni obliczeń dla
      każdego wierzchołka do osobnego wątku. Użycie większej ilości wątków znacząco
      przyspieszyło algorytm (\hyperref[tab:speedup]{Tabela \ref*{tab:speedup}}).

      \subsubsection{Preprocessing} Ze względu na możliwy niespójny zbiór indeksów
      wierzchołków, w celu zachowania liniowej pamięci oraz dostępu do elementów
      struktur pomocniczych w czasie $\mathcal{O}(1)$, zastosowany został
      preprocessing przydzielający każdemu wierzchołkowi unikalny numer
      $order \in \left[0, |V|\right)$. Preprocessing wykonywany jest podczas tworzenia
      wierzchołka i nie zwiększa złożoności czasowej budowania grafu $\mathcal{O}(|E|\log|V|)$.

      \subsubsection{Stała liczba instancji} Aby przy dużej liczbie wierzchołków nie
      tworzyć wielu tymczasowych struktur do obliczeń, klasa \texttt{Scheduler} dba o
      stworzenie dokładnie tylu instancji \texttt{BrandesScope} ile może zostać
      użytych wątków. Każdorazowo gdy instancja \texttt{BrandesScope} dostaje
      nowy wierzchołek nie następuje alokacja nowej pamięci, a jedynie wyczyszczenie
      wcześniejszych struktur i rozpoczęcie obliczeń.

  \newpage
  \section{Statystyki}
    \subsection{Przeprowadzenie pomiarów}
    Dla każdego \texttt{m = 1, 2, \ldots, 8} uruchomiono 100-krotnie program
    \texttt{brandes} poleceniem
    \begin{verbatim}
      $ time ./brandes m wiki-vote-sort.txt wiki.out
    \end{verbatim}
    \vspace{-3ex}
    a wyniki zapisano do plików. Za szukane wartości $p\textsubscript{m}$ przyjęto
    średnią arytmetyczną z otrzymanych wyników.

    \subsection{Środowisko pomiarowe}
    Testy zostały przeprowadzone dla zestawu danych \texttt{wiki-vote-sort.txt}.
    Do pomiarów wykorzystano:
    \begin{enumerate}[label={(\arabic*)}]
      \item
        \label{test:first}
        komputer osobisty wyposażony w 8-rdzeniowy procesor Intel\textsuperscript{
        \textregistered} Core\texttrademark i7 4700HQ oraz dysk ADATA\textsuperscript{
        \textregistered} Solid State Drive.
      \item
        \label{test:second}
        serwer \texttt{students.mimuw.edu.pl}
    \end{enumerate}

    \begin{table}[h]
      \centering
      \label{tab:times:first}
      \begin{tabular}{|c|c|c|c|}
        \hline
        \bfseries $m$ & \bfseries minimalny czas & \bfseries maksymalny czas &
        \bfseries średni czas ($p\textsubscript{m}$)\\
        \hline
        1 & 2,12  & 2,647 & 2,192 \\
        2 & 1,148 & 1,704 & 1,261 \\
        3 & 0,82  & 1,035 & 0,887 \\
        4 & 0,641 & 0,781 & 0,682 \\
        5 & 0,614 & 0,758 & 0,63  \\
        6 & 0,594 & 0,69  & 0,617 \\
        7 & 0,583 & 0,656 & 0,614 \\
        8 & 0,572 & 0,646 & 0,612 \\
        \hline
      \end{tabular}
      \caption{Czas wykonania \hyperref[test:first]{(Środowisko 1)}}
    \end{table}

    \begin{table}[h]
      \centering
      \label{tab:times:second}
      \begin{tabular}{|c|c|c|c|}
        \hline
        \bfseries $m$ & \bfseries minimalny czas & \bfseries maksymalny czas &
        \bfseries średni czas ($p'\textsubscript{m}$)\\
        \hline
        1 & 4,906 & 9,902 & 7,527 \\
        2 & 3,261 & 6,463 & 4,754 \\
        3 & 2,332 & 4,517 & 3,483 \\
        4 & 2,029 & 3,53  & 2,722 \\
        5 & 1,622 & 3,036 & 2,274 \\
        6 & 1,515 & 2,49  & 1,933 \\
        7 & 1,26  & 2,308 & 1,823 \\
        8 & 1,189 & 2,069 & 1,562 \\
        \hline
      \end{tabular}
      \caption{Czas wykonania \hyperref[test:second]{(Środowisko 2)}}
    \end{table}
    \subsection{Przyspieszenie}
    Przyjęto oznaczenia jak w poprzednich tabelach -- $p\textsubscript{m}$ dla
    \hyperref[test:first]{środowiska 1} oraz $p'\textsubscript{m}$ dla
    \hyperref[test:second]{środowiska 2}.
    \begin{table}[h]
      \centering
      \label{tab:speedup}
      \begin{tabular}{|c|c|c|c|c|}
        \hline
        \bfseries $m$ & \bfseries $p\textsubscript{m}$ & \bfseries $p'\textsubscript{m}$
        & \bfseries $s\textsubscript{m} = p\textsubscript{1}/p\textsubscript{m}$
        & \bfseries $s'\textsubscript{m} = p'\textsubscript{1}/p'\textsubscript{m}$ \\
        \hline
        1 & 2,192 & 7,527 & 1       & 1       \\
        2 & 1,261 & 4,754 & 1,7383  & 1,5833  \\
        3 & 0,887 & 3,483 & 2,47125 & 2,16107 \\
        4 & 0,682 & 2,722 & 3,21408 & 2,76525 \\
        5 & 0,63  & 2,274 & 3,47937 & 3,31003 \\
        6 & 0,617 & 1,933 & 3,55267 & 3,89395 \\
        7 & 0,614 & 1,823 & 3,57003 & 4,12891 \\
        8 & 0,612 & 1,562 & 3,5817  & 4,81882 \\
        \hline
      \end{tabular}
      \caption{Przyspieszenie}
    \end{table}

\end{document}
