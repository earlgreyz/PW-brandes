\documentclass{article}
\usepackage{polski}
\usepackage[utf8]{inputenc}
\usepackage{hyperref}
\hypersetup{
    colorlinks,
    linkcolor=[rgb]{.0 .152 .645}
}
\usepackage{enumitem}
\usepackage{listings}
\usepackage[edges]{forest}
\definecolor{folderbg}{RGB}{124,166,198}
\definecolor{folderborder}{RGB}{110,144,169}
\newlength\Size
\setlength\Size{4pt}
\tikzset{%
  folder/.pic={%
    \filldraw [draw=folderborder, top color=folderbg!50, bottom color=folderbg] (-1.05*\Size,0.2\Size+5pt) rectangle ++(.75*\Size,-0.2\Size-5pt);
    \filldraw [draw=folderborder, top color=folderbg!50, bottom color=folderbg] (-1.15*\Size,-\Size) rectangle (1.15*\Size,\Size);
  },
  file/.pic={%
    \filldraw [draw=folderborder, top color=folderbg!5, bottom color=folderbg!10] (-\Size,.4*\Size+5pt) coordinate (a) |- (\Size,-1.2*\Size) coordinate (b) -- ++(0,1.6*\Size) coordinate (c) -- ++(-5pt,5pt) coordinate (d) -- cycle (d) |- (c) ;
  },
}
\forestset{%
  declare autowrapped toks={pic me}{},
  pic dir tree/.style={%
    for tree={%
      folder,
      font=\ttfamily,
      grow'=0,
    },
    before typesetting nodes={%
      for tree={%
        edge label+/.option={pic me},
      },
    },
  },
  pic me set/.code n args=2{%
    \forestset{%
      #1/.style={%
        inner xsep=2\Size,
        pic me={pic {#2}},
      }
    }
  },
  pic me set={directory}{folder},
  pic me set={file}{file},
}

\title{Współbieżny algorytm Brandesa}
\author{Mikołaj Walczak}

\begin{document}
  \pagenumbering{gobble}
  \maketitle
  \newpage

  \tableofcontents
  \listoffigures
  \newpage

  \pagenumbering{arabic}
  \section{Implementacja}
    Poniższy rozdział zawiera jedynie krótki opis implementacji. Szczegóły
    dotyczące klas i metod przez nie udostępnianych można znaleźć w
    \hyperref[documentation]{dokumentacji technicznej}.
    \label{implementation}
    \subsection{Podział na pliki}
      \label{files}
      Implementacja została podzielona względem odpowiedzialności za poszczególne
      części programu na pliki jak na obrazku poniżej (\hyperref[fig:files]
      {Rysunek \ref*{fig:files}}).

      \begin{figure}[h]
        \label{fig:files}
        \begin{forest}
          pic dir tree, where level=0{}{directory,},
          [brandes
            [src
              [betweennes.h/.cpp \textrm{-- zawiera implementację algorytmu
                Brandesa}, file]
              [graph.h/.cpp \textrm{--
                klasy \hyperref[class:brandes__node]{\texttt{Node}}
                oraz \hyperref[class:brandes__graph]{\texttt{Graph}}}
              , file]
              [scheduler.h \textrm{--
                klasy \hyperref[class:synchronization__scope]{\texttt{Scope}}
                oraz \hyperref[class:synchronization__scheduler]
                {\texttt{Scheduler}}}
              , file]
              [main.cpp \textrm{-- odpowiedzialny za wykonanie programu}, file]
            ]
          ]
        \end{forest}
        \caption{Podział na pliki}
      \end{figure}

    \subsection{Podział na klasy}
      Podział na klasy przedstawiony jest na \hyperref[fig:classes]
      {Rysunku \ref*{fig:classes}}.
      \label{classes}
      \begin{figure}[h]
        \label{fig:classes}
        \begin{forest}
          for tree={
            font=\ttfamily,
            grow'=0,
            child anchor=west,
            parent anchor=south,
            anchor=west,
            calign=first,
            edge path={
              \noexpand\path [draw, \forestoption{edge}]
              (!u.south west) +(7.5pt,0) |- node[fill,inner sep=1.25pt] {} (.child anchor)\forestoption{edge label};
            },
            before typesetting nodes={
              if n=1
                {insert before={[,phantom]}}
                {}
            },
            fit=band,
            before computing xy={l=15pt},
          }
          [
            [namespace \texttt{\hyperref[namespace:brandes]{Brandes}}
              [class \texttt{\hyperref[class:brandes__node]{Node}}]
              [class \texttt{\hyperref[class:brandes__graph]{Graph}}]
              [function \texttt{\hyperref[func:brandes__calculate_weights]
                {calculate\_weights}}]
            ]
            [namespace \texttt{\hyperref[namespace:synchronization]
              {Synchronization}}
              [class \texttt{\hyperref[class:synchronization__scope]{Scope}}]
              [class \texttt{\hyperref[class:synchronization__scheduler]
                {Scheduler}}]
            ]
          ]
        \end{forest}
        \caption{Podział na klasy}
      \end{figure}

      \subsubsection{Przestrzenie nazw}
        \label{namespaces}

        \paragraph{Brandes}
        \label{namespace:brandes}
        zawiera klasy i funkcje związane bezpośrednio z implementacją algorytmu
        Brandesa.

        \paragraph{Synchronization}
        \label{namespace:synchronization}
        zawiera klasy i funkcje zapewniające współbieżność wykonywanych obliczeń.


      \subsubsection{Brandes::Node}
      \label{class:brandes__node}
      Reprezentuje pojedynczy wierzchołek w grafie skierowanym. Pozwala na
      bezpieczną wielowątkowo zmianę wagi przez operację
      \texttt{increase\_weight(weight)}.

      \subsubsection{Brandes::Graph}
      \label{class:brandes__graph}
      Reprezentuje graf skierowany. Umożliwia wczytanie sktrutury grafu ze strumienia
      \texttt{std::istream} oraz zapisanie wierzchołków wraz z wagami do strumienia
      \texttt{std::ostream}.

      \subsubsection{Brandes::calculate\_weights}
      \label{func:brandes__calculate_weights}
      Implementacja algorytmu Brandesa. Liczy wagi wierzchołków dla danego
      \hyperref[class:brandes__graph]{\texttt{grafu}} tworząc obiekt klasy
      \hyperref[class:synchronization__scheduler]{\texttt{Scheduler}} z podaną
      liczbą wątków i zleca mu współbieżne obliczenie wag osobno dla każdego
      wierzchołka. Dzięki gwarancji bezpieczeństwa metody
      \texttt{increase\_weight(weight)} klasy \hyperref[class:brandes__node]
      {\texttt{Node}} obliczenia mogą występować jednocześnie dla różnych
      wierzchołków.

      \subsubsection{Synchronization::Scope}
      \label{class:synchronization__scope}
      Klasa abstrakcyjna (właść. interface), z którego musi dziedziczyć klasa
      środowiska obliczeniowego \hyperref[class:synchronization__scheduler]
      {\texttt{Scheduler}}'a. Dla danego zadania wykonuje wirtualną funkcję
      \texttt{execute(task)}.

      \subsubsection{Synchronization::Scheduler}
      \label{class:synchronization__scheduler}
      Główny mechanizm synchronizujący. Tworzy pulę wątków a w każdym z nich
      instancję podanej klasy \hyperref[class:synchronization_scope]{\texttt{Scope}}.
      Zleca wątkom wykonywanie funkcji \texttt{execute(task)} dla każdego otrzymanego
      zadania.

    \subsection{Dokumentacja techniczna}
      \label{documentation}
      Do wygenerowania dokumentacji wymagana jest pakiet Doxygen
      w wersji $\geq$ 1.8.8. Wykonanie poniższych instrukcji w katalogu
      projektu spowoduje wygenerowanie dokumentacji w folderach jak na
      \hyperref[fig:documentation]{Rysunku \ref*{fig:documentation}}.
      \begin{verbatim}
        $ mkdir build && cd $_
        $ cmake ..
        $ make doc
      \end{verbatim}
      \begin{figure}[h]
        \label{fig:documentation}
        \begin{forest}
          pic dir tree, where level=0{}{directory,},
          [brandes
            [build
              [doc
                [html \textrm{-- dokumentacja w wersji HTML}]
                [latex \textrm{-- dokumentacja w wersji \LaTeX}]
              ]
              [Doxyfile, file]
              [Makefile, file]
            ]
            [CMakeLists.txt \textrm{-- instrukcje budowy dokumentacji}, file]
            [Doxyfile.in \textrm{-- konfiguracja dokumentacji}, file]
            [doxygen \textrm{-- pliki stylu dokumentacji HTML}]
            [src \textrm{-- kod źródłowy}]
          ]
        \end{forest}
        \caption{Generowanie dokumentacji}
      \end{figure}

  \section{Optymalizacje}
    \begin{itemize}[noitemsep]
    \item Dostęp w czasie $\mathcal{O}(1)$ do elementów w tablicach dzięki
    preprocessingowi w czasie $\mathcal{O}(V)$.
    \item Uwspółbieżnienie kroku dla każdego wierzchołka
    \item Zachowanie liniowej pamięci dzięki użyciu Schedulera i BrandesScope
    \end{itemize}
  \newpage
  \section{Statystyki}

\end{document}
