\documentclass{article}
\usepackage{polski}
\usepackage[utf8]{inputenc}
\usepackage{hyperref}
\hypersetup{
    colorlinks,
    linkcolor=[rgb]{.0 .152 .645}
}
\usepackage{enumitem}
\usepackage{listings}
\usepackage[edges]{forest}
\definecolor{folderbg}{RGB}{124,166,198}
\definecolor{folderborder}{RGB}{110,144,169}
\newlength\Size
\setlength\Size{4pt}
\tikzset{%
  folder/.pic={%
    \filldraw [draw=folderborder, top color=folderbg!50, bottom color=folderbg] (-1.05*\Size,0.2\Size+5pt) rectangle ++(.75*\Size,-0.2\Size-5pt);
    \filldraw [draw=folderborder, top color=folderbg!50, bottom color=folderbg] (-1.15*\Size,-\Size) rectangle (1.15*\Size,\Size);
  },
  file/.pic={%
    \filldraw [draw=folderborder, top color=folderbg!5, bottom color=folderbg!10] (-\Size,.4*\Size+5pt) coordinate (a) |- (\Size,-1.2*\Size) coordinate (b) -- ++(0,1.6*\Size) coordinate (c) -- ++(-5pt,5pt) coordinate (d) -- cycle (d) |- (c) ;
  },
}
\forestset{%
  declare autowrapped toks={pic me}{},
  pic dir tree/.style={%
    for tree={%
      folder,
      font=\ttfamily,
      grow'=0,
    },
    before typesetting nodes={%
      for tree={%
        edge label+/.option={pic me},
      },
    },
  },
  pic me set/.code n args=2{%
    \forestset{%
      #1/.style={%
        inner xsep=2\Size,
        pic me={pic {#2}},
      }
    }
  },
  pic me set={directory}{folder},
  pic me set={file}{file},
}

\title{Współbieżny algorytm Brandesa}
\author{Mikołaj Walczak}

\begin{document}
  \pagenumbering{gobble}
  \maketitle
  \newpage

  \tableofcontents
  \listoffigures
  \newpage

  \pagenumbering{arabic}
  \section{Implementacja}
    \subsection{Podział na pliki}
    Implementacja została podzielona względem odpowiedzialności za poszczególne części
    programu na następujące pliki (Rysunek \ref{fig:files}).

    \begin{figure}[h]
      \begin{forest}
        pic dir tree, where level=0{}{directory,},
        [brandes
          [src
            [betweennes.h/.cpp \textrm{-- zawiera implementację algorytmu Brandesa}, file]
            [graph.h/.cpp \textrm{--
              klasy \hyperref[class:brandes__node]{\texttt{Brandes::Node}}
              oraz \hyperref[class:brandes__graph]{\texttt{Brandes::Graph}}}
            , file]
            [scheduler.h \textrm{--
              klasy \hyperref[class:synchronization__scope]{\texttt{Synchronization::Scope}}
              oraz \hyperref[class:synchronization__scheduler]{\texttt{Synchronization::Scheduler}}}
            , file]
            [main.cpp \textrm{-- odpowiedzialny za wykonanie programu}, file]
          ]
        ]
      \end{forest}
      \caption{Podział na pliki}
      \label{fig:files}
    \end{figure}

    \subsection{Podział na klasy}
    \label{class:brandes__node}
    Brandes::Node
    \label{class:brandes__graph}
    Brandes::Graph
    \label{class:synchronization__scope}
    Synchronization::Scope
    \label{class:synchronization__scheduler}
    Synchronization::Scheduler

    \subsection{Dokumentacja techniczna}
    Do wygenerowania dokumentacji wymagana jest instalacja pakietu Doxygen w wersji $\geq$ 1.8.8.
    Wykonanie poniższych instrukcji spowoduje wygenerowanie dokumentacji w katalogu \texttt{build/doc}.
    \begin{verbatim}
      $ cd brandes
      $ mkdir build && cd $_
      $ make doc
    \end{verbatim}

    \begin{forest}
      pic dir tree, where level=0{}{directory,},
      [brandes
        [build
          [doc
            [html \textrm{-- dokumentacja w wersji HTML}]
            [latex \textrm{-- dokumentacja w wersji \LaTeX}]
          ]
          [Doxyfile, file]
          [Makefile, file]
        ]
        [CMakeLists.txt \textrm{-- instrukcje budowy dokumentacji}, file]
        [Doxyfile.in \textrm{-- konfiguracja dokumentacji}, file]
        [doxygen \textrm{-- pliki stylu dokumentacji}]
        [src \textrm{-- kod źródłowy}]
      ]
    \end{forest}


  \section{Optymalizacje}
    \begin{itemize}[noitemsep]
    \item Dostęp w czasie $\mathcal{O}(1)$ do elementów w tablicach dzięki preprocessingowi
    w czasie $\mathcal{O}(V)$.
    \item Uwspółbieżnienie kroku dla każdego wierzchołka
    \item Zachowanie liniowej pamięci dzięki użyciu Schedulera i BrandesScope
    \end{itemize}
  \newpage
  \section{Statystyki}

\end{document}
