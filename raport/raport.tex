\documentclass{article}
\usepackage{polski}
\usepackage[utf8]{inputenc}
\usepackage{enumitem}
\usepackage{listings}
\usepackage[edges]{forest}
\definecolor{folderbg}{RGB}{124,166,198}
\definecolor{folderborder}{RGB}{110,144,169}
\newlength\Size
\setlength\Size{4pt}
\tikzset{%
  folder/.pic={%
    \filldraw [draw=folderborder, top color=folderbg!50, bottom color=folderbg] (-1.05*\Size,0.2\Size+5pt) rectangle ++(.75*\Size,-0.2\Size-5pt);
    \filldraw [draw=folderborder, top color=folderbg!50, bottom color=folderbg] (-1.15*\Size,-\Size) rectangle (1.15*\Size,\Size);
  },
  file/.pic={%
    \filldraw [draw=folderborder, top color=folderbg!5, bottom color=folderbg!10] (-\Size,.4*\Size+5pt) coordinate (a) |- (\Size,-1.2*\Size) coordinate (b) -- ++(0,1.6*\Size) coordinate (c) -- ++(-5pt,5pt) coordinate (d) -- cycle (d) |- (c) ;
  },
}
\forestset{%
  declare autowrapped toks={pic me}{},
  pic dir tree/.style={%
    for tree={%
      folder,
      font=\ttfamily,
      grow'=0,
    },
    before typesetting nodes={%
      for tree={%
        edge label+/.option={pic me},
      },
    },
  },
  pic me set/.code n args=2{%
    \forestset{%
      #1/.style={%
        inner xsep=2\Size,
        pic me={pic {#2}},
      }
    }
  },
  pic me set={directory}{folder},
  pic me set={file}{file},
}

\title{Współbieżny algorytm Brandesa}
\author{Mikołaj Walczak}

\begin{document}
  \pagenumbering{gobble}
  \maketitle

  \newpage
  \pagenumbering{arabic}
  \section{Implementacja}
  Dokładny opis implementacji można znaleźć w dokumentacji technicznej.
  \subsection{Dokumentacja techniczna}
  Do wygenerowania dokumentacji wymagana jest instalacja pakietu Doxygen w wersji $\geq$ 1.8.8.
  Wykonanie poniższych instrukcji spowoduje wygenerowanie dokumentacji w katalogu \texttt{build/doc}.
  \begin{verbatim}
    $ cd brandes
    $ mkdir build && cd $_
    $ make doc
  \end{verbatim}
  \begin{forest}
    pic dir tree,
  where level=0{}{% folder icons by default; override using file for file icons
    directory,
  },
    [brandes
      [build
        [doc
          [html \textrm{dokumentacja w wersji HTML}]
          [latex \textrm{dokumentacja w wersji \LaTeX}]
        ]
        [Doxyfile, file]
        [Makefile, file]
      ]
      [CMakeLists.txt \textrm{instrukcje budowy dokumentacji}, file]
      [Doxyfile.in \textrm{konfiguracja dokumentacji}, file]
      [doxygen \textrm{pliki stylu dokumentacji}]
      [src \textrm{kod źródłowy}]
    ]
  \end{forest}


  \section{Optymalizacje}
    \begin{itemize}[noitemsep]
    \item Dostęp w czasie $\mathcal{O}(1)$ do elementów w tablicach dzięki preprocessingowi
    w czasie $\mathcal{O}(V)$.
    \item Uwspółbieżnienie kroku dla każdego wierzchołka
    \item Zachowanie liniowej pamięci dzięki użyciu Schedulera i BrandesScope
    \end{itemize}
  \newpage
  \section{Statystyki}

\end{document}
